\chapter*{\centering\Large{Tóm tắt đề tài}}
\addcontentsline{toc}{chapter}{Tóm tắt đề tài}
Kiểm thử xâm nhập là một trong những phương pháp phổ biến nhất để đánh giá bảo mật của một hệ thống, ứng dụng hoặc mạng. Mặc dù đã có những công cụ hỗ trợ với hiệu quả tương đối cao, nó vẫn thường được thực hiện thủ công và hầu hết dựa vào kinh nghiệm bởi các \textit{hacker} có đạo đức, được gọi là pentester. 

Nhằm phát triển một phương pháp mới có thể vừa tận dụng được khả năng tổng hợp và tính toán nhanh, mạnh của máy tính, đồng thời dựa trên kinh nghiệm của pentester để không bỏ lọt các lỗ hổng, đề tài giới thiệu một phương pháp kiểm thử xâm nhập tự động sử dụng học tăng cường sâu (RL). Thông qua agent RL được huấn luyện với mô hình A3C, tổng hợp kinh nghiệm để chọn một 
payload chính xác để khai tác lỗ hổng có sẵn.

Kết quả của phương pháp là một công cụ có khả năng thu thập thông tin, khai thác lỗ hổng hiện có, đồng thời báo cáo kết quả thu được. Sau khi được huấn luyện với các tham số môi trường, agent RL có thể hỗ trợ pentester nhanh chóng xác định các lỗ hổng. Phương pháp này giúp giảm thiểu các vấn đề về chi phí lao động, thời gian chuẩn bị dữ liệu cho quá trình tự động kiểm thử xâm nhập.

Các kết quả mang lại ở bước đầu tương đối tích cực đã chứng minh rằng agent RL có thể tổng hợp và phân tích các kết quả từ môi trường trước đó, khai thác thành công cho các môi trường tiếp theo ngay trong lần thử đầu tiên.



