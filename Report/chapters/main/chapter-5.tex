\chapter{Tổng kết}
\label{chapter5}
\section{Kết quả đạt được}
Trong nghiên cứu này, chúng tôi đã khám phá hiệu suất của DRL trong xây dựng một công cụ kiểm thử xâm nhập tự động thông qua Metasploit Framework để thực hiện quét và khai thác kèm tích lũy kinh nghiệm. 

Chúng tôi cũng đã xây dựng các môi trường lỗ hổng và thực hiện các kịch bản thử nghiệm khác nhau để đánh giá toàn diện hiệu suất của công cụ, bao gồm đào tạo và thử nghiệm dựa trên hiệu quả khai thác đạt được. Trong quá trình thử nghiệm, công cụ của chúng tôi đã có thể khai thác thành công tất cả các lỗ hổng dịch vụ mà chúng tôi đã tạo trong môi trường khai thác lý tưởng. 

Với kết quả xuất sắc này, công cụ này đã chứng minh được khả năng tích lũy kết quả học từ các môi trường trước để thành công khai thác lỗ hổng cho lần khai thác tiếp theo trong môi trường khác ngay lần đầu tiên. 

Tuy nhiên, môi trường chúng tôi thử nghiệm là môi trường lý tưởng nhất, khả năng khai thác của công cụ có thể giảm đi nếu môi trường chứa tường lửa hoặc số cổng bị thay đổi. 

\section{Hướng phát triển}
Dựa vào kết quả đạt được, chúng tôi nhận thấy công cụ có tiềm năng phát triển trong tương lai, tuy nhiên cần phải có nhiều chỉnh sửa hơn nữa để có thể đạt được  hiệu quả học tập tốt nhất. Chúng ta có thể chuẩn bị thêm nhiều môi trường chứa lỗ hổng hơn để huấn luyện và tạo ra sự đa dạng về kinh nghiệm khai thác cho công cụ, cung cấp môi trường hoạt động cho các tác nhân học song song độc lập với nhau, giải quyết nhược điểm đụng độ khai thác. 

Bên cạnh đó, chúng ta cũng có thể mở rộng chức năng công cụ thêm ở các bước khác trong quy trình kiểm thử thâm nhập, hoặc sử dụng thêm hậu khai thác để tiếp tục tấn công vào các máy chủ mục tiêu khác trong mạng cục bộ của nó. 

Ngoài ra, chúng ta có thể tạo thêm mô-đun phụ trợ cho công cụ để có thể sử dụng các mô-đun khai thác ở các mức xếp hạng khác nhau trên Metasploit, từ đó mở rộng khả năng khai thác của công cụ. Đây là một công việc phức tạp và đòi hỏi thời gian rất lớn, bởi các mô-đun khai thác (Mức xếp hạng normal) khác nhau sẽ có những thông tin cấu hình riêng cho phù hợp với các lỗ hổng dịch vụ khác nhau.